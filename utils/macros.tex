\newcommand{\jroedit}[2]{\sout{#1}{\color{red}{#2}}}
% \newcommand{\jroedit}[2]{#2}
\newcommand{\jrocomment}[1]{({\color{pgreen}{JRO's comment:}} \textbf{\color{pgreen}{#1}})}

\newcommand{\citationNeeded}{\textcolor{magenta}{\textbf{[CITATION NEEDED]}}\xspace}
\newcommand{\tableNeeded}{\textcolor{magenta}{\textbf{[TABLE NEEDED]}}\xspace}
\newcommand{\figureNeeded}{\textcolor{magenta}{\textbf{[FIGURE NEEDED]}}\xspace}
\newcommand{\highLight}[1]{\textcolor{magenta}{\MakeUppercase{#1}}}

\newcommand{\fig}{Figure\xspace}
\newcommand{\figs}{Figures\xspace}
\newcommand{\tbl}{Table\xspace}
\newcommand{\tbls}{Tables\xspace}

\newcommand{\datasets}{data sets\xspace}
\newcommand{\dataset}{data set\xspace}

\newcommand{\ignore}[1]{}
\newcommand{\addTail}[1]{\textit{#1}.---}
\newcommand{\super}[1]{\ensuremath{^{\textrm{#1}}}}
\newcommand{\sub}[1]{\ensuremath{_{\textrm{#1}}}}
\newcommand{\dC}{\ensuremath{^\circ{\textrm{C}}}}
\newcommand{\tn}{\tabularnewline}
\newcommand{\spp}[1]{\textit{#1}}

\providecommand{\e}[1]{\ensuremath{\times 10^{#1}}}

\newcommand{\change}[2]{{\color{red} #2}\xspace}
\newcommand{\thought}[1]{\textcolor{purple}{THOUGHT: #1}}

\newcommand{\widthFigure}[5]{\begin{figure}[htbp]
\begin{center}
    \includegraphics[width=#1\textwidth]{#2}
    \captionsetup{#3}
    \caption{#4}
    \label{#5}
    \end{center}
    \end{figure}}

\newcommand{\heightFigure}[5]{\begin{figure}[htbp]
\begin{center}
    \includegraphics[height=#1\textheight]{#2}
    \captionsetup{#3}
    \caption{#4}
    \label{#5}
    \end{center}
    \end{figure}}

\newcommand{\smartFigure}[5]{%
    \begin{figure}[htbp]
        \begin{center}
            \includegraphics[width=\textwidth,height=#1\textheight,keepaspectratio]{#2}
            \captionsetup{#3}
            \caption{#4}
            \label{#5}
        \end{center}
    \end{figure}
}

\newcommand{\mFigure}[4]{\smartFigure{#1}{#2}{listformat=figList}{#3}{#4}\clearpage}
\newcommand{\embedHeightFigure}[4]{\heightFigure{#1}{#2}{listformat=figList}{#3}{#4}}
\newcommand{\embedWidthFigure}[4]{\widthFigure{#1}{#2}{listformat=figList}{#3}{#4}}
\newcommand{\siFigure}[4]{\smartFigure{#1}{#2}{name=Figure S, labelformat=noSpace, listformat=sFigList}{#3}{#4}\clearpage}

%% macro to make long strings breakable over lines
\makeatletter
\def\breakable#1{\xHyphen@te#1$\unskip}
\def\xHyphen@te{\@ifnextchar${\@gobble}{\sw@p{\allowbreak{}\xHyphen@te}}}
% \def\xHyphen@te{\@ifnextchar${\@gobble}{\sw@p{\hskip 0pt plus 1pt\xHyphen@te}}}
\def\sw@p#1#2{#2#1}
\makeatother

\newcommand{\accuracyscatterplotannotations}[1]{For each plot, the
    root-mean-square error (RMSE) and the proportion of estimates for which the
    95\% credible interval contained the true value---$p(#1 \in
    \textrm{CI})$---is given.
}
\newcommand{\neventsplotannotations}{For each plot,
    the proportion of simulation replicates for which the number of events with
    the largest posterior probability matched the true number of
    events---$p(\hat{\nevents} = \nevents)$---is shown in the upper left
    corner,
    the median posterior probability of the correct number of events across all
    simulations---$\widetilde{p(\nevents|\alldata)}$---is shown in the upper
    right corner, and
    the proportion of simulations for which the true divergence model was
    included in the 95\% credible set---$p(\nevents \in
    \textrm{CS})$---is shown in the lower right.
}
\newcommand{\neventsbarplotannotations}{For each plot, the following summary
    statistics are shown:
    the proportion of simulation replicates for which the number of events with
    the largest posterior probability matched the true number of
    events---$p(\hat{\nevents} = \nevents)$---,
    the median posterior probability of the correct number of events across all
    simulations---$\widetilde{p(\nevents|\alldata)}$---, and
    the proportion of simulations for which the true divergence model was
    included in the 95\% credible set---$p(\nevents \in
    \textrm{CS})$---.
}
\newcommand{\neventsshadingdescription}{The number of simulation replicates
    that fall within each possible cell of true versus estimated numbers of
    events is shown, and cells with more replicates are shaded darker.
}
